\documentclass[]{TUD_CS_FYP_Report}

% TABLE OF CONTENTS STUFF
\usepackage{tocloft}
\setcounter{tocdepth}{4}
\setcounter
{secnumdepth}{4}
\renewcommand{\cftsecfont}{\normalfont}
\renewcommand{\cftsubsecfont}{\normalfont}
\renewcommand{\cftsubsubsecfont}{\normalfont}
\renewcommand{\cftsecleader}{\cftdotfill{\cftdotsep}}
\renewcommand{\cftsubsecleader}{\cftdotfill{\cftdotsep}}
\renewcommand{\cftsubsubsecleader}{\cftdotfill{\cftdotsep}}

% COLOURS FOR LINKS AND CITATIONS ETC
\definecolor{tud_blue}{RGB}{5,77,109}
\definecolor{light_tud_blue}{RGB}{67,132,161}
\definecolor{dark_tud_blue}{RGB}{2,50,71}
\definecolor{light_grey}{gray}{0.9707}

\hypersetup{
    colorlinks=true,
    linkcolor=tud_blue,
    filecolor=light_tud_blue,      
    urlcolor=light_tud_blue,
    citecolor=light_tud_blue
}

% CODE LISTING CONFIG

% REFERENCE STYLE
\usepackage[backend=biber,style=vancouver]{biblatex}
\addbibresource{references.bib}

%%%%%%%%%%%%%%%%%%%%%%%%%%%%%%%%%%%%%%%%%%%%%%%%%%%
                % BEGIN DOCUMENT %                
%%%%%%%%%%%%%%%%%%%%%%%%%%%%%%%%%%%%%%%%%%%%%%%%%%%

\def\studentname{Your Name}         % Edit with your name
\def\studentid{C12345678}           % Edit with your student number
\def\projecttitle{Your Project Title} % Edit with you project title
\def\supervisorname{Dr. Supervisor} % Edit with your supervisor

\begin{document}
\maketitle

\abstract
Summary of your project

\chapter*{Declaration}
I hereby declare that the work described in this dissertation is, except where otherwise stated, entirely my own work and has not been submitted as an exercise for a degree at this or any
other university.

\vspace{0.5cm}
\begin{flushright}
    \textit{Your Name}\\
    \vspace{0.25cm}
    \today
\end{flushright}
\pagebreak

\chapter*{Acknowledgements}
Thank whoever

\pagebreak
\tableofcontents
\pagebreak
\listoffigures
\pagebreak

\chapter{Tutorial}
\section{Introductions}
This is the best template you can use for your FYP (in my opinion). Do \textbf{NOT} use Word, use \LaTeX\ instead and here's why:

\begin{itemize}
    \item Looks pretty.
    \item Citations and referencing sections in your report are 100x easier.
    \item You can move things around and keep things consistent.
    \item Code listings are cool looking.
\end{itemize}

It seems intimidating at first but it isn't too bad once you get the hang of things.

\section{Citations}
Here is me citing a paper from \texttt{references.bib} \cite{achiam2023gpt}. You can get BibTeX from Google Scholar easily. You can also do footnotes like this.\footnote{\url{https://example.com/}}

\section{Tables}
This is just a table style I like to use.

\begin{table}[H]
    \centering
    \rowcolors{2}{white}{light_grey}  % Alternate row colours
    \begin{tabular}{l l l}
        \toprule
        \textbf{Header 1} & \textbf{Header 2} & \textbf{Header 3} \\
        \midrule
        data & more data & even more data \\
        data & more data & even more data \\
        data & more data & even more data \\
        data & more data & even more data \\
        data & more data & even more data \\
        \bottomrule
    \end{tabular}
    \caption{Cool Table}
    \label{table:1}
\end{table}

\section{Code Listings}
 Listing \ref{lst:follow-me} is shown below.

\begin{figure}[H]
\centering
\fontsize{11}{11}\selectfont
\begin{minted}
[frame=lines,framesep=5mm,baselinestretch=1.2,bgcolor=light_grey,linenos,style=friendly,breaklines,
label={main.py}
]{python}
from typing import Optional

def follow_me(github: Optional[str] = "https://github.com/ronan-s1") -> None:
    print(f"FOLLOW ME: {github}")

if __name__ == "__main__":
    follow_me()
\end{minted}
\vspace{-0.6cm}
\captionsetup{type=listing}
\caption{Python Code Listing}
\label{lst:follow-me}
\end{figure}
\vspace{-0.43cm}

You can change the style and language simply as shown in Listing \ref{lst:follow-me-2}.

\begin{figure}[H]
\centering
\fontsize{11}{11}\selectfont
\begin{minted}
[frame=lines,framesep=5mm,baselinestretch=1.2,bgcolor=light_grey,linenos,style=emacs,breaklines,
label={main.ml}
]{ocaml}
let follow_me ?(github : string option) () =
  let github_url =
    match github with
    | Some username -> "https://github.com/" ^ username
    | None -> "https://github.com/ronan-s1"
  in
  print_endline ("FOLLOW ME: " ^ github_url)

let () =
  follow_me ()
\end{minted}
\vspace{-0.6cm}
\captionsetup{type=listing}
\caption{OCaml Code Listing}
\label{lst:follow-me-2}
\end{figure}
\vspace{-0.43cm}

\section{Inserting Images}
You can see a dangerous cat in Figure \ref{fig:dangerous-cat}. You can insert images by pasting them in too.

\begin{figure}[H]
    \centering
    \includegraphics[width=0.25\linewidth]{images/cat.png}
    \caption{Dangerous Cat}
    \label{fig:dangerous-cat}
\end{figure}

\section{This is a Section}
\subsection{This is a Sub Section}
\subsubsection{This is a Sub Sub Section}

% BIBLIOGRAPHY
\printbibliography
\end{document}
